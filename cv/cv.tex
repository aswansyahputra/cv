%!TEX TS-program = xelatex
%!TEX encoding = UTF-8 Unicode
% Awesome CV LaTeX Template for CV/Resume
%
% This template has been downloaded from:
% https://github.com/posquit0/Awesome-CV
%
% Author:
% Claud D. Park <posquit0.bj@gmail.com>
% http://www.posquit0.com
%
%
% Adapted to be an Rmarkdown template by Mitchell O'Hara-Wild
% 23 November 2018
%
% Template license:
% CC BY-SA 4.0 (https://creativecommons.org/licenses/by-sa/4.0/)
%
%-------------------------------------------------------------------------------
% CONFIGURATIONS
%-------------------------------------------------------------------------------
% A4 paper size by default, use 'letterpaper' for US letter
\documentclass[11pt, a4paper]{awesome-cv}

% Configure page margins with geometry
\geometry{left=1.4cm, top=.8cm, right=1.4cm, bottom=1.8cm, footskip=.5cm}

% Specify the location of the included fonts
\fontdir[fonts/]

% Color for highlights
% Awesome Colors: awesome-emerald, awesome-skyblue, awesome-red, awesome-pink, awesome-orange
%                 awesome-nephritis, awesome-concrete, awesome-darknight

\definecolor{awesome}{HTML}{414141}

% Colors for text
% Uncomment if you would like to specify your own color
% \definecolor{darktext}{HTML}{414141}
% \definecolor{text}{HTML}{333333}
% \definecolor{graytext}{HTML}{5D5D5D}
% \definecolor{lighttext}{HTML}{999999}

% Set false if you don't want to highlight section with awesome color
\setbool{acvSectionColorHighlight}{true}

% If you would like to change the social information separator from a pipe (|) to something else
\renewcommand{\acvHeaderSocialSep}{\quad\textbar\quad}

\def\endfirstpage{\newpage}

%-------------------------------------------------------------------------------
%	PERSONAL INFORMATION
%	Comment any of the lines below if they are not required
%-------------------------------------------------------------------------------
% Available options: circle|rectangle,edge/noedge,left/right

\name{Muhammad Aswan \textbf{Syahputra}}{}

\position{Data Analyst · Sensometrics Specialist}
\address{Puri Cipageran Indah I Blok A No 191, Kota Cimahi, Jawa Barat, Indonesia}

\mobile{(+62) 822 3465 3816}
\email{\href{mailto:muhammadaswansyahputra@gmail.com}{\nolinkurl{muhammadaswansyahputra@gmail.com}}}
\github{aswansyahputra}
\linkedin{aswansyahputra}

% \gitlab{gitlab-id}
% \stackoverflow{SO-id}{SO-name}
% \skype{skype-id}
% \reddit{reddit-id}

\quote{I would like to recognize myself as a Food Data Scientist. I have an expertise in Sensory Science with a Food Technology background, a passion in data analysis and visualisation, an eagerness in R statistical programming, and a hobby in developing analytical web application.}

\usepackage{booktabs}

% Templates for detailed entries
% Arguments: what when with where why
\usepackage{etoolbox}
\def\detaileditem#1#2#3#4#5{%
\cventry{#1}{#3}{#4}{#2}{\ifx#5\empty\else{\begin{cvitems}#5\end{cvitems}}\fi}\ifx#5\empty{\vspace{-4.0mm}}\else\fi}
\def\detailedsection#1{\begin{cventries}#1\end{cventries}}

% Templates for brief entries
% Arguments: what when with
\def\briefitem#1#2#3{\cvhonor{}{#1}{#3}{#2}}
\def\briefsection#1{\begin{cvhonors}#1\end{cvhonors}}

\providecommand{\tightlist}{%
	\setlength{\itemsep}{0pt}\setlength{\parskip}{0pt}}

%------------------------------------------------------------------------------



\begin{document}

% Print the header with above personal informations
% Give optional argument to change alignment(C: center, L: left, R: right)
\makecvheader

% Print the footer with 3 arguments(<left>, <center>, <right>)
% Leave any of these blank if they are not needed
% 2019-02-14 Chris Umphlett - add flexibility to the document name in footer, rather than have it be static Curriculum Vitae
\makecvfooter
  {February 2020}
    {Muhammad Aswan \textbf{Syahputra}~~~·~~~Curriculum Vitae}
  {\thepage}


%-------------------------------------------------------------------------------
%	CV/RESUME CONTENT
%	Each section is imported separately, open each file in turn to modify content
%------------------------------------------------------------------------------



\hypertarget{work}{%
\section{Work}\label{work}}

\detailedsection{\detaileditem{Data analyst}{Nov 2019 – Present}{Jabar Digital Service}{Bandung, Indonesia}{\item{Gather insight from sectoral data to support decision and policy making}\item{Optimise the budget and resource allocation within government }\item{Improve the quality of digital products that tailored for government and public use}}\detaileditem{Sensometrics specialist}{Nov 2018 – Present}{Sensolution.ID – Food Sensory and Sensometrics Consulting}{Bandung, Indonesia}{\item{Develop a software for sensory data collection and analyses.}\item{Deliver training to clients for advance sensory methods and analyses.}\item{Assist clients to arrange, design, and perform sensory research.}}\detaileditem{Instructor}{May 2019 – Present}{R Academy of Telkom University}{Bandung, Indonesia}{\item{Design syllabus for R training courses.}\item{Deliver training for R courses, including: introduction to R for data science, data visualisation in R, machine learning in R, fundamental tidy data science in R.}}\detaileditem{Intern}{Apr 2018 – Jul 2018}{Sensory and Applied Food Research Group at Universitas Brawijaya}{Malang, Indonesia}{\item{Developed prototype of a new software for sensory data analyses.}\item{Delivered lectures to students in sensory evaluation and methods course.}\item{Conducted research to compare users’ experience between two different softwares when performing sensory data analyses.}}\detaileditem{Independent freelance}{2014 – Present}{Research Consultancy and Data Analysis Services}{Indonesia – The Netherlands}{\item{Assisted clients to properly choose and design an appropriate food research.}\item{Assisted clients to learn and analyse data collected from food research.}\item{Developed an analytical software to help clients analysing data.}}\detaileditem{Intern}{Dec 2013 – Feb 2014}{Indonesian Coffee and Cocoa Research Institute (ICCRI)}{Jember, Indonesia}{\item{Assisted and prepared the procedure to evaluate sensory characteristic of specialty coffee.}\item{Assisted sensory evaluation of chocolate products}}\detaileditem{Assistant for laboratory practicum}{2012 – 2015}{Faculty of Agricultural Technology, Universitas Brawijaya}{Malang, Indonesia}{\item{Prepared syllabus for laboratory practicum.}\item{Assisted, supervised, and evaluated students performance during laboratory works.}\item{Laboratory of Sensory Analysis, Laboratory of Post-harvest Processing Technology, and Laboratory of Chemistry.}}}

\hypertarget{education}{%
\section{Education}\label{education}}

\detailedsection{\detaileditem{Master in Food Technology, Sensory Science specialisation}{2016-2018}{Wageningen University and Research}{Wageningen, The Netherlands}{\item{Thesis: “Observation of Chewing Activity using Photoplethysmography (PPG) method”. Supervisors: Monica Mars and Chirstos Diou.}\item{Internship: “Development of SenseHub: an integrated web application for sensory analyses”. Supervisors: Kiki Fibrianto and Betina Piqueras-Fiszman.}\item{Excelled in: Advance Sensory Science and Sensometrics (MCB-32806), Integrated Sensory Science (MCB-33306), Instrumental Sensory Science (HNE-30606), Product and Process Design (FQD-60312), and Advance Statistics (MAT-20306).}\item{Granted with Indonesian Education Scholarship from Indonesia Endowment Fund of Education (LPDP).}\item{CGPA of 7.53/10.00.}}\detaileditem{Bachelor in Agricultural Product Technology, Food Science and Technology program}{2011-2015}{Universitas Brawijaya}{Malang, Indonesia}{\item{Thesis: “Exploratory study: the influence of iced-coffee drinking methods to consumers’ multisensory perception utilizing Rate-all-that-Apply (RATA) method”. Supervisors: Kiki Fibrianto and Indria Purwantiningrum.}\item{Internship: “Sensory evaluation of specialty coffee using cups tasting procedure”. Supervisors: Kiki Fibrianto and Yusianto.}\item{Excelled in: Food Product Development and Sensory Evaluation, Food Processing Technology, and Statistic and Computation.}\item{Granted with PPA Scholarship from Ministry of Research, Technology and Higher Education of Indonesia.}\item{CGPA of 3.80/4.00.}}}

\hypertarget{external-activity}{%
\section{External Activity}\label{external-activity}}

\detailedsection{\detaileditem{Initiator}{2016 – Present}{Komunitas R Indonesia}{Indonesia}{\item{Komunitas R Indonesia is a non-profit community for Indonesian to learn and share knowledge about R statistical programming language. I initiated this community back in 2016 and now it has more than 1000 members came from cities in Indonesia. Aside from managerial activity, in several occasions I also gave presentation and training to the members during community meetup. Additionally, I share my knowledge about R programming in Youtube video in hope that it can be beneficial to broader audience.}}\detaileditem{Coordinator}{Sep 2016 – Sep 2017}{Association of Indonesian Education Scholarship Awardee in Wageningen University and Research}{Wageningen, The Netherlands}{\item{The massive number of Indonesian scholarship awardee in Wageningen University and Research introduces challenge in communication, sharing of information, and management between students and Indonesian government. I took an initiative and an opportunity to help both students and Indonesian government to face these challenges. Furthermore, I also assisted the students who had issues or problems in studying, financial, administration, and daily life.}}\detaileditem{Member and Data Analyst}{2015 – Present}{Food Sensory Research Group Universitas Brawijaya}{Malang, Indonesia}{\item{Food Sensory Research Group was the first student group in Universitas Brawijaya that focused in sensory science and research. In this group we constantly have discussion about new issues in sensory science, advance sensory methods, how to handle complexity in data, and practical applications of sensory science in Food Technology domain.}}\detaileditem{Tutor for Students}{2012 – 2014}{Himpunan Mahasiswa Teknologi Hasil Pertanian (HIMALOGISTA) Universitas Brawijaya}{Malang, Indonesia}{\item{I volunteered to help and assist students in studying several courses taught in Food Science and Technology department, including: Fundamentals of Food Process Engineering, Food Process Engineering, Research Methodology and Experimental Design, and Statistics and Computation.}}}

\hypertarget{research}{%
\section{Research}\label{research}}

\detailedsection{\detaileditem{JOIV : International Journal on Informatics Visualization}{2019}{Thermostats: an Open Source Shiny App for Your Open Data Repository}{Padang, Indonesia}{\item{Researchers: Dasapta Erwin Irawan, Muhammad Aswan Syahputra, Prana Ugi, Deny Juanda Puradimaja}\item{In this research we collected water quality from 416 geothermal sites across Indonesia and built an analytical, data-driven, and user-focused online application for analyzing geothermal water quality}}\detaileditem{Kuala Lumpur International Agriculture, Forestry \& Plantation Conference (KLIAFP)}{2018}{Consumer Preference and Satisfaction Analysis on Croissant in Malang Area}{Selangor, Malaysia}{\item{Researchers: Jaya Mahar Maligan, Yuke Pamelasari, Muhammad Aswan Syahputra.}\item{In this research we investigated various external product properties that may give contribution to consumer preference and satisfaction. By using External Preference Mapping method, we were able to discover the perceptual mapping of various Croissant brands and the segmentation of consumers group.}}\detaileditem{International Coference on Green Agroindustry and Bioeconomic (ICGAB)}{2018}{SenseHub: an integrated web application for sensory analyses}{Malang, Indonesia}{\item{Researchers: Muhammad Aswan Syahputra, Kiki Fibrianto, Eka Shinta Wulandari.}\item{In this research we compared the users’ experience when using generic statistical software versus SenseHub for performing sensory data analyses. SenseHub is a web application that was developed by the author to make sensory data analysis easier while embedded with plenty of state-of-the-art sensometrics procedures.}}\detaileditem{Master Thesis at Wageningen University and Research}{2018}{Observation of Chewing Activity using Photoplethysmography (PPG) Method}{Wageningen, The Netherlands}{\item{Researchers: Muhammad Aswan Syahputra, Monica Mars, Christos Diou.}\item{In this research we developed algorithms to monitor chewing activity during eating solid and semi-solid foods using a new sensor, Photoplethsmography (PPG). We measured the performance of the proposed algorithms and made comparison with the existing procedures for monitoring chewing, i.e. video recording analysis and Electro Myography (EMG) method.}}\detaileditem{Jurnal Pangan dan Agroindustri}{2016}{Effects of Delivery Methods to Multisensory Perceptions: a Review}{Malang, Indonesia}{\item{Researchers: Muhammad Aswan Syahputra, Kiki Fibrianto, Indria Purwantiningrum.}\item{In this concise review article we were discussing the possible effect of delivery methods of food and drink to multisensory perceptions during eating.}}\detaileditem{Bachelor Thesis at Universitas Brawijaya}{2015}{Exploratory Study: the Influence of Instant Iced-Coffee Drinking Methods to Consumers' Multisensory Perceptions Utilizing Rate-all-that-Apply (RATA) Method}{Malang, Indonesia}{\item{Researchers: Muhammad Aswan Syahputra, Kiki Fibrianto, Indria Purwantiningrum.}\item{In this research we asked participants to consume iced-coffee in three ways: drinking using straw; immediate drinking from the glass; and tasting using spoon. We discovered that the difference in drinking method may alter consumers’ multisensory perception during drinking.}}}

\hypertarget{delivered-presentation-and-training}{%
\section{Delivered Presentation and Training}\label{delivered-presentation-and-training}}

\detailedsection{\detaileditem{Komunitas R Indonesia meetup at School of Business and Management, Bandung Institute of Technology}{31 August 2019}{Extending RStudio with git and GitHub}{Bandung, Indonesia}{\item{Introduction to version control system, benefit of using git and GitHub for data science project, how to integrate RStudio with git and GitHub, workflow of using git and GitHub from RStudio}}\detaileditem{Komunitas R Indonesia meetup at Sadasa Academy}{16 August 2019}{Data rectangling in R: a journey from JSON to CSV}{Yogyakarta, Indonesia}{\item{Data structure in R, subsetting elements of data object, functional iteration in R, importing JSON to R, transforming JSON structure into tabular data object}}\detaileditem{Public workshop at R Academy Telkom University}{26-27 July 2019}{Fundamental tidy data science in R}{Bandung, Indonesia}{\item{Introduction to Tidyverse for tackling data science problems, machine learning principle and workflow, case study supervised learning: predicting numbers and classes, case study unsupervised learning: topic modeling for texts, building a web application for interactive data visualization, deploying machine learning models into a web application}}\detaileditem{Internal workshop at School of Business and Management, Bandung Institute of Technology}{5 July 2019}{Translating consumer voice for market insight: an introduction to sensory science}{Bandung, Indonesia}{\item{Sensory science and its domain, application of sensory science, simulation of sensory research, demonstration of sensory data analysis}}\detaileditem{Public workshop at R Academy Telkom University}{20-24 May 2019}{Bootcamp R for data science}{Bandung, Indonesia}{\item{R essentials, working with version control system, data science workflow using tidyverse in R, machine learning and statistics, and case study of common machine learning problems}}\detaileditem{Komunitas R Indonesia meetup at Algoritma Education Center}{11 May 2019}{Data carpentry with tidyverse}{Jakarta, Indonesia}{\item{Definition of data carpentry for handling data, common functions for data transformation, scrapping and cleaning data from webpage}}\detaileditem{Public workshop at R Academy Telkom University}{13 April 2019}{Mastering data visualisation}{Bandung, Indonesia}{\item{Introduction to Grammar of Graphics, building a graph layer by layer, creating various types of data visualisation and customisation, manipulating graph looks and feels}}\detaileditem{Public workshop at R Academy Telkom University}{6 April 2019}{R introduction for data science}{Bandung, Indonesia}{\item{R fundamentals, importing data from various sources, performing data manipulation, creating data visualisation, and building models and iteration}}\detaileditem{Komunitas R Indonesia meetup at Perbanas Institute}{23 March 2019}{Let’s build your first RStudio snippets and addins}{Jakarta, Indonesia}{\item{Make use of RStudio snippets to write common codes, introduction to addins, writing custom code snippets, and creating an R package containing addins}}\detaileditem{Komunitas R Indonesia meetup at Telkom University}{22 February 2019}{Introduction to R (+) for data science}{Bandung, Indonesia}{\item{Introduction to R, empowering R with RStudio+git, glimpse of tidyverse packages collection, chaining multiple operations in R, and practical activity on analysing Indonesia presidential election data}}\detaileditem{Datascientalk HIMATIKA Bandung Institute of Technology}{26 January 2019}{R (+) for data science}{Bandung, Indonesia}{\item{R as statistical programming language, RStudio as IDE, git+GitHub as version control system, and tidyverse for data science activity}}\detaileditem{Komunitas R Indonesia meetup at Microsoft Indonesia}{19 January 2019}{R+RStudio tips and tricks:hidden gems to improve workflow}{Jakarta, Indonesia}{\item{Trivia in R statistical programming language, guidelines for coding style, advices on how to setup R project, helpful RStudio shortcuts and tricks to improve workflow}}\detaileditem{Internal workshop at Department of Food Science and Technology, Bogor Agricultural University}{19-20 December 2019}{Workshop for sensory data analyses}{Bogor, Indonesia}{\item{Conventional and advance sensory methods, statistical approach for analysing sensory data, and practical on how to analyse sensory data from various sensory methods using R and SenseHub}}\detaileditem{Komunitas R Indonesia meetup at Bandung Institute of Technology}{15 November 218}{R in food sensory research}{Bandung, Indonesia}{\item{Application of R for linking food properties and consumer preference, food production optimisation, and detection of chewing activity}}\detaileditem{International Coference on Green Agroindustry and Bioeconomic (ICGAB)}{19 September 2018}{SenseHub: an integrated web application for sensory analyses}{Malang, Indonesia}{\item{Research methods in sensory science, discussion of available software for sensory analyses, proposal of new sensory software named SenseHub and its comparison with existing statistical software }}}

\hypertarget{developed-software}{%
\section{Developed Software}\label{developed-software}}

\briefsection{\briefitem{Syahputra, MA. (2019). jabr: an R interface for accessing Open Data Jawa Barat.}{\textbf{R package}}{}\briefitem{Syahputra, MA. (2019). sensehubr: a tidy tools for sensory data analysis.}{\textbf{R package}}{}\briefitem{Syahputra, MA. (2019). bandungjuara: an R interface for accessing Open Data Kota Bandung.}{\textbf{R package}}{}\briefitem{Syahputra, MA. (2019). jakartaraya: an R interface for accessing Open Data Kota Jakarta.}{\textbf{R package}}{}\briefitem{Syahputra, MA. (2019). nusandata: a curated datasets about Indonesia.}{\textbf{R package}}{}\briefitem{Syahputra, MA. (2019). SenseQuest: an web application for sensory data collection.}{\textbf{Web apps}}{}\briefitem{Syahputra, MA. (2018). SenseHub: an integrated web application for sensory analysis.}{\textbf{Web apps}}{}\briefitem{Irawan, DE., Syahputra, MA. (2018). Aquastats: a web application for groundwater data analyses.}{\textbf{Web apps}}{}\briefitem{Irawan, DE., Syahputra, MA. (2018). Thermostats: a web application for hot water data analyses.}{\textbf{Web apps}}{}\briefitem{Syahputra, MA. (2018). MCDM: a shiny application for performing Multiple Criteria Decision Making (MCDM).}{\textbf{Web apps}}{}}

\end{document}
